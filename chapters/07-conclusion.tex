\chapter{Conclusion}\label{chp:conclusion}
In our study, we set out to answer three research questions: 
\begin{itemize}
    \item Are convolutional neural networks able to accurately classify the images according to the different classes ?
    \item Which network performs best ?
    \item How does initialization and research parameters influence the performance of the network ?
\end{itemize}


The results of this study is that we can predict on the porosity class with xx\% accuracy, the Dunham class with XX\%, DRT with XX\% and the components with a F1-sore of xx\%. Considering the subjectivity that exist with this labeling as we explained in \ref{chp:discussion}, we can consider that those results are highly satisfying. We can confirm that CNN can be used for this classification task.
For the second question, we found that Inception\_v3 or GoogLeNet was the best performing network. This is surprising since ResNet is the best state of the art network today. 
Finally, we found  that depending on each network, the initialization achieved very different results. For ResNet and AlexNet, the pretrained weights helped achieve better and quicker results. Whereas GoogLeNet performed best with random initialization. Also, we used a different optimizer and a longer training time to see how far we could take Inception\v3. The use of Stochastic Gradient Descent with xxx as a learning rate gave us the best results.

The aim of this project was to build a classification algorithm that could be used by geologists to help them in their day to day labeling work. We can say that this goal was reach since we have a robust and accurate network, able to classify the different classes. We should of course keep in mind that the aim is not to replace the geologist on site, but to provide them with first assumption that they can chose to confirm or challenge. 

In order to take this project further, we would need more data to train our model on. Once this is used by geologists, the feedback that we will receive is going to improve the quality of our prediction. In a few years, the geologists might only have a checking task when labeling core samples. The algorithm would output such goo prediction that no further analysis work is required.  
