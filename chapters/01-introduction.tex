\chapter{Introduction}
\label{chp:introduction} 
In this chapter we explain the motivation for this thesis and state our goals and research questions. Further, we  proceed to explain how we plan to answer our research question by introducing our research methods.   
\section{Motivation}
In spite of the shift towards renewable energies, the search for oil and gas remains as strong as ever. To remain competitive with the low oil price, operators need to find new ways to optimize their services. 
With increasing amount of data,  more and more companies are embracing Machine Learning for their day to day operations. And the results are encouraging. 
From drilling to stock prices, the industry researches new tools to help automate some of the processes. 
Still, geologists and other experts remain reluctant to trust these "black box algorithms". This is why we need to strive to achieve better results and convince them that Machine Learning can assist them in their daily work.

Last year, several newspapers argued whether "data [was] now more valuable than oil" \cite{economist,forbes}. Even though opinion differ, there is a consensus that data drive innovation and scientific progress. This highlights the need for the oil industry to become more digital and find ways to draw value from this data. A good example of how data can help is carbonate rocks classification \cite{carbo}. Dozens of different data sources are exploited such as seismic data, data from various tools and sensors while drilling, pictures of rocks, etc. If  we could analyze all these data better, we would be able to recover more oil from the fields we are already exploiting, and discover even bigger ones. The use of deep learning in petroleum studies is still emerging, but the possibilities are countless.


The goal of this thesis is to produce an image recognition algorithm that is able to differentiate between different levels of porosity,  Dunham classification, components and depositional rock types using picture of thin sections of core rocks. The purpose is to help geologists labeling carbonate rocks since it can be a time consuming, laborious and rather expensive task. Experienced geologist describe around 10 thin sections per day when they go to a moderate level of detail. 

This algorithm is built with the help of \gls{cnn}. We feed the networks with images of thin sections labeled by geologists and it learns to identify the different elements stated earlier. We consider that we reached the aim when we produce an algorithm capable of efficiently predict the amount of porosity, Dunham classification, depositional rock type and correct components in a new images we feed it. 

\section{Research Questions}
To reach this goal, we define a general research question \textbf{RQ}. 
\begin{itemize}
    \item \textbf{RQ}: How efficiently can \gls{cnn} classify the quality of reservoir rocks based on pictures of thin sections ?
\end{itemize}

In order to answer these question, we define three sub-questions: 
\begin{itemize}
    \item Are Convolutional Neural Networks able to accurately classify the images according to the different classes ?
    \item Which network performs best ?
    \item How does initialization and hyperparameters influence the performance of the network ?
\end{itemize}

\section{Disclaimer}
This thesis has been conducted in collaboration with a company. The input data itself is confidential. This is why it has been anonymized. We refer to the different components with letters and depositional rock types with codes. Apart from that, the results are fully published here.  

\section{Research Methods}\label{sec:research-method}
To reach the goal and address the research question, we divide our research into two parts. First, we conduct a literature review to understand the labels we are trying to identify and the theory behind \gls{cnn} and their different optimization. In the second part we will test different networks and optimization techniques. From a viewpoint of scientific methodology, the research method we follow in the second part is quantitative. Methodology will be discussed further in Chapter \ref{chp:methodology}.

\section{Thesis Outline}

In Chapter \ref{chp:introduction}, we introduced the motivations for this thesis. It contains arguments for how an image classification algorithm could facilitate the job of geologist and allow them to focus on tasks that need more expertise. Further the aim of the thesis and the related research questions were presented. 


Chapter \ref{chp:background} contains the theory needed to understand the data given and how the labels are associated to the quality of the reservoir rocks.It also presents the theory needed to build an image recognition algorithm based on \gls{cnn}. More specifically, it introduces the different layers and operations that they consist of, as well as optimization processes. Lastly, the architecture and theory behind the state of the art network and the different techniques used to train them are explained.
 

In Chapter \ref{chp:methodology}, we introduce the methods used in the image classification algorithm. In the first part, we discuss the label selection and the preprocessing. Then we describes the data augmentation techniques. In a third section, we look into details into the architecture of each network and lastly we present the different experiments that were conducted.  


Chapter \ref{chp:results} simply presents the results of each experiment for each labels using tables, plots and confusion matrices.


In Chapter \ref{chp:discussion}, we discuss the results and the eventual limitations that we have identified.


In Chapter \ref{chp:conclusion}, we draw conclusions from the results and discuss thoughts about future work and improvements that could be conducted. 
